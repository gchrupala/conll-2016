\subsection{Prediction of visual features}
For the experiments on the prediction of visual features all models
were trained on the training set of MS COCO. As validation data we
used a random sample of 5000 images from the MS COCO validation set. 

Figure~\ref{fig:loss} shows the value of the validation average cosine distance
between the predicted visual vector and the target vector for three
random initializations of each of the model types. 

The Phonetic GRU model is more sensitive to the initialization: one
can clearly distinguish three separate trajectories. The word-level models
are much less affected by random initialization. In terms of the
overall performance, the {\sc Phoneme GRU} model falls between the
{\sc Word Sum} model and th {\sc Word GRU} model.

\begin{figure}
  \centering
  \includegraphics[scale=0.3]{loss-zoom.pdf}
  \caption{Value of the loss function on validation data during
    training. Three random initialization of each model are shown.}
  \label{fig:loss}
\end{figure}

We also evaluated the models on how well they perform when used to
search images: for each validation sentence the model was used to predict the
visual vector. The image vectors in the validation data were then
ranked by cosine similarity to the predicted vector, and the
proportion of times the correct image was among the top 3 was
reported. By {\it correct} image we mean the one which the sentence
was used to describe (even though often many other images are also
good matches to the sentence). 

In Figure~\ref{fig:accat5} we report the accuracies on this task for
the two word-level models, as well as for the Phoneme GRU model with
different number of hidden layers. We trained each model version with
three random initializations for each model setting, and evaluate
after each epoch. We report the score of the best epoch for each
initialization. The overall ranking of the models matches the direct
evaluation of the loss function above: the phoneme-level models are in
between the two word-level models. Phoneme GRU with three hidden
layers is the best of the phoneme-level model. This model achieved the
maximum score after 8 epochs. In subsequent experiments we use these
particular model parameters when reporting other results with {\sc
  Phoneme GRU}.

\begin{figure}
  \centering
  \includegraphics[scale=0.25]{accat5.pdf}
  \caption{Accuracy at 5 on the image retrieval task}
  \label{fig:accat5}
\end{figure}

% \begin{table}
%   \centering
%   \begin{tabular}{ll|r}
%  Model     & Layers & Acc. at 5 \\\hline
%  Word Sum  & 0      & 0.158 \\
%  Word GRU  & 1      & 0.205 \\\hline
%  Phon GRU  & 1      & 0.169  \\
%  Phon GRU  & 2      & 0.176  \\
%  Phon GRU  & 3      & \bf 0.183  \\  
%  Phon GRU  & 4      & 0.179 \\
%  Phon GRU  & 5      & 0.170 \\
%   \end{tabular}
% \caption{Accuracy at 5 on the image retrieval task.}
% \label{tab:accat5}
% \end{table}
