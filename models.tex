\section{Models}
\label{sec:models}

The architecture of our model is depicted in Figure
\ref{fig:architecture} and consists of a phoneme embedding layer,
followed by a stack of $N$ Gated Recurrent Neural nets, followed by
densely connected layer which maps the last hidden state of the top
recurrent layer to a vector of visual features. 

Gated Recurrent Units (GRU) were introduced in
\newcite{cho2014properties} and \newcite{chung2014empirical} as an
attempt to alleviate the problem of vanishing gradient in standard
simple recurrent nets as known since the work of
\newcite{elman1990finding}. GRUs have a linear shortcut through
timesteps which bypasses the nonlinearity and thus promotes gradient
flow.
Specifically, a GRU computes the hidden state at current time step, $\mathbf{h}_{t}$, as the
linear combination of previous activation $\mathbf{h_{t-1}}$, and a new
{\it candidate} activation $\mathbf{\tilde{h}}_t$:
%

\begin{equation}
  \mathrm{gru}(\mathbf{h}_{t-1}, \mathbf{x}_t) = (1 - \mathbf{z}_t)\odot \mathbf{h}_{t-1} + \mathbf{z}_t \odot \mathbf{\tilde{h}}_t
\vspace{-.1cm}
\end{equation}
%
where $\odot$ is elementwise multiplication, and the update gate
activation $\mathbf{z_{t}}$ determines the amount of new information
mixed in the current state:
%

\begin{equation}
\label{eq:gru-update}
   \mathbf{z}_t = \sigma_s(\mathbf{W}_z \mathbf{x}_t + \mathbf{U}_z \mathbf{h}_{t-1})
\end{equation}
%
The candidate activation is computed as:
%
\begin{equation}
\label{eq:gru-cand}
   \mathbf{\tilde{h}}_t = \sigma(\mathbf{W} \mathbf{x}_t + \mathbf{U}(\mathbf{r}_t \odot \mathbf{h}_{t-1}))
\end{equation}
%
The reset gate $\mathbf{r_{t}}$ determines how much of the current
input $\mathbf{x_{t}}$ is mixed in the previous state
$\mathbf{h}_{t-1}$ to form the candidate activation:
%
\begin{equation}
\label{eq:gru-reset}
   \mathbf{r}_t = \sigma_s(\mathbf{W}_r \mathbf{x}_t + \mathbf{U}_r \mathbf{h}_{t-1})
\end{equation}

By applying the $\mathrm{gru}$ function repeatedly a GRU layer maps a
sequence of inputs to a sequence of states:
\begin{equation}
  \mathrm{GRU}(\mathbf{X}, \mathbf{h}_0) = \mathrm{gru}(\mathbf{x}_n, \dots, \mathrm{gru}(\mathbf{x}_2, \mathrm{gru}(\mathbf{x}_1, \mathbf{h}_0)))
\end{equation}
