\documentclass[11pt]{article}
\usepackage{coling2016}
\usepackage[utf8]{inputenc}
\usepackage{times}
\usepackage{url}
\usepackage{latexsym}
\usepackage{amsmath}
\usepackage{amsfonts}
\usepackage{graphicx}
\usepackage{tipa}
%\emnlpfinalcopy

%\setlength\titlebox{5cm}

% How about this:
\title{From phonemes to images: levels of representation in a recurrent neural
model of visually-grounded language learning}
 % \author{Lieke Gelderloos \\
 %   Tilburg University \\
 %   {\tt l.j.gelderloos@uvt.nl} \And
 %   Grzegorz Chrupała \\
 %   Tilburg University \\
 %   {\tt g.chrupala@uvt.nl} }
 \author{Author 1 \And Author 2}


\date{}

\begin{document}
\maketitle
\begin{abstract}
We present a model of visually-grounded language learning based on stacked gated recurrent neural networks which learns to predict visual features given an image description in the form of a sequence of phonemes. The learning task resembles that faced by human language learners who need to discover both structure and meaning from noisy and ambiguous data across modalities. We show that our model indeed learns to predict features of the visual context given phonetically transcribed image descriptions, and show that it represents linguistic information in a hierarchy of levels: lower layers in the stack are comparatively more sensitive to form, whereas higher layers are more sensitive to meaning.
\end{abstract}


\section{Introduction}
\label{sec:intro}

According to \newcite{Chrupala2015LearningLT} this is like that \cite{Kdr2016RepresentationOL}.
\newcite{DBLP:journals/corr/RibeiroSG16} introduce blah.
\section{Related work}
A large part of language acquisition consists of learning its structure, but in order to be able to communicate it is also of vital importance to learn the relation of words to entities in the world. % I don't like this sentence
Grounded lexical acquisition is often modeled as cross-situational learning, a process of rule-based \cite{siskind.96} or statistical inference \cite{fazly.etal.10csj, frank.etal.07} of word-to-referent mappings, based on the co-occurence of words and objects over multiple encounters. 
Cross-situational models typically work on word-level language input and symbolic representations of the context, whereas infants have to learn from continuous perceptual input. Recent experimental and computational studies have found that co-occurring visual information may help to learn word forms \cite{thiessen2010effects, Cunillera2010295, Glicksohn2013, Yurofsky2012statistical}. This suggests that acquisition of word form and meaning should be seen as interactive, rather than separate processes.

The Cross-channel Early Lexical Learning (CELL) model of \newcite{Roy2002113} and the more recent work of \newcite{rasanen2015joint} take into account the continuous nature of the speech signal, and incorporate visual information as well. The CELL model learns to discover words in continuous speech through co-occurence with their visual referent, but the visual input only consists of the shape of single objects, effectively bypassing referential uncertainty. \newcite{rasanen2015joint} propose a probabilistic joint model of word segmentation and meaning acquisition from raw speech and a set of possible referents that appear in the context. In both \newcite{Roy2002113} and \newcite{rasanen2015joint}, then, the visual context is considerably less noisy and ambiguous than that available to children.

There is an extensive line of artificial intelligence research on automatic image captioning (see \newcite{bernardi2016automatic} for a recent overview). Typically, image captioning models learn to recognize high-level image features and associate them with words. Inspired by both image captioning research and cross-situational human language acquisition, two recent automatic speech recognition models learn to recognize word forms from visual data. In \newcite{synnaeve2014learning}, language input consists of single spoken words and visual data consists of image fragments, which the model learns to associate. \newcite{harwath2015deep} employ two convolutional neural networks, a visual object recognition model and a word recognition model, and an embedding alignment model that learns to map recognized words and objects into the same high-dimensional space. Although the object recognition works on the raw visual input, the speech signal is segmented into words before presenting it to the word recognition model. Both \newcite{harwath2015deep} and \newcite{synnaeve2014learning} recognize words from pre-segmented speech, but the location of word boundaries is not available to language learning infants.

The character level has recently gained attention in various NLP applications. \newcite{ling2015finding} and \newcite{plank2016multilingual} use bidirectional LSTMs that read every word character-by-character and produce composed word vectors. While these word vectors have the advantage of encoding information about the surface form and can even be informative for unknown words, they can only be formed when word boundaries are known. % terrible no good very bad sentence. 
\newcite{chung2016character} present machine translation with character level output, but the \textit{input} consists of sub-word units as in \newcite{sennrich2015neural}. These units may correspond to whole words, morphemes, or characters, but are restricted to subsequences of individual words and never cross word boundaries. Character-level neural NLP \textit{without} explicit exploitation of word boundaries in the input is studied in some specific cases where fixed vocabularies are inherently problematic, e.g. with combined natural and programming language input \cite{chrupala2013text} or when specifically dealing with misspelled words in automatic writing feedback \cite{xie2016neural}. 

Character-level language models are described and interpreted in \newcite{hermans2013training} and \newcite{karpathy2015visualizing}. Both studies show that character-level deep recurrent neural networks are sensitive to long-range dependencies: for example by keeping track of opening and closing parentheses over long stretches of text. \newcite{hermans2013training} describe the hierarchical organization that seems to emerge during training, with higher layers processing information over longer timescales. In our work we show related effects in the context of a model of visually-grounded language learning from unsegmented phonetic strings. Even when they are not exploited explicitly, whitespaces and punctuation still provide implicit cues to the location of word boundaries. Our input data does not contain these cues.

In the current study we use phonetic transcription of full sentences as a first step towards large-scale multimodal language learning from speech co-occurring with visual scenes. In contrast to \newcite{Roy2002113} and \newcite{rasanen2015joint}, the visual input to our model consists of high-level visual features, which means it contains ambiguity and noise. In contrast to \newcite{synnaeve2014learning} and \newcite{harwath2015deep}, we consider full utterances rather than separate words. As \newcite{harwath2015deep} note, the learning task of a multimodal model becomes significantly more complicated when the language input consists of unsegmented speech. The absence of word boundaries in the input data is essential to our aim, and we choose to use unsegmented phoneme transcriptions rather than raw speech. Additionally, phonemic input data allows us to perform experiments on the encoding of linguistic knowledge as reported in section \ref{sec:experiments} without additional annotation.

To our knowledge, there is no work yet on multimodal phoneme or character-level language modeling with visual input. % multimodal and visual - double?
Our study is complementary to work on character-level language modeling: while language models focus on {\it within}-text co-occurrence statistics as clues to language structure, in our setting the weight is on the correlations  {\it between} language and the visual scenes. 
Although the language input in this study is low-level-symbolic rather than perceptual, the learning problem we aim to solve is similar to that of a human language learner: the learner has to discover language structure as well as meaning, based on ambiguous and noisy data from another modality. 

\newcite{chrupala2015learning} simulate visually grounded human language learning with specific sensitivity to the noise and ambiguity in the visual domain. They model language learning as a task of predicting visual context given a sequence of words. While the visual input consists of a continuous representation, the language input consists of a sequence of words. The aim of this study is to take their approach one step further towards multimodal language learning from raw perceptual input. 

\newcite{Kdr2016RepresentationOL} develop techniques for understanding, interpretation and visualization of the representations of linguistic form and meaning in recurrent neural networks, and apply these to word-level models. In our work we share the goal of revealing the nature of emerging representations, but we do not assume words and word embeddings as their basic unit. Also, we are especially concerned with the emergence of a hierarchy of levels of representations in stacked recurrent networks.
\section{Models}
Consider a learner who sees a person pointing at a scene and uttering
the unfamiliar phrase
{\it Look, the monkeys're playing}. We may suppose that the learner
will update her language understanding model such that the
subsequent utterance of this phrase will evoke in her mind something
close to the impression of this visual scene. Our model is a particular
instantiation of this simple idea.
\subsection{Phon GRU}
The architecture of our main model of interest, {\sc Phon GRU} is
schematically depicted in Figure \ref{fig:architecture} and consists of a phoneme
encoding  layer, followed by a stack of $K$ Gated Recurrent Neural
nets, followed by a densely connected layer which maps the last hidden
state of the top recurrent layer to a vector of visual features.

\begin{figure}
\begin{minipage}[l]{0.45\textwidth}
  \includegraphics[scale=0.2]{architecture.pdf}
  \caption{A three-timestep slice of the stacked recurrent architecture with three hidden layers.}
  \label{fig:architecture}
\end{minipage}
\hspace{0.3cm}
\begin{minipage}[r]{0.45\textwidth}
  \begin{tabular}{|l|}\hline
    \includegraphics[scale=0.7]{woman-ipa.png} \\
    A young woman riding a horse holding a flag\\\hline
  \end{tabular}
  \begin{center}
    \includegraphics[scale=0.2]{rider.jpg}
  \end{center}

  \caption{(Top) Example of a postprocessed phonetic transcription
    from eSpeak used as input to the {\sc Phon GRU} model. (Bottom)
    Corresponding image.}
  \label{fig:ipa}
\end{minipage}
\end{figure}

Gated Recurrent Units (GRU) were introduced in
\newcite{cho2014properties} and \newcite{chung2014empirical} as an
attempt to alleviate the problem of vanishing gradient in standard
simple recurrent nets as known since the work of
\newcite{elman1990finding}. GRUs have a linear shortcut through
timesteps which bypasses the nonlinearity and thus promotes gradient
flow.
Specifically, a GRU computes the hidden state at current time step, $\mathbf{h}_{t}$, as the
linear combination of previous activation $\mathbf{h_{t-1}}$, and a new
{\it candidate} activation $\mathbf{\tilde{h}}_t$:
%

\begin{equation}
  \mathrm{gru}(\mathbf{x}_t, \mathbf{h}_{t-1}) = (1 - \mathbf{z}_t)\odot \mathbf{h}_{t-1} + \mathbf{z}_t \odot \mathbf{\tilde{h}}_t
\vspace{-.1cm}
\end{equation}
%
where $\odot$ is elementwise multiplication, and the update gate
activation $\mathbf{z_{t}}$ determines the amount of new information
mixed in the current state:
%

\begin{equation}
\label{eq:gru-update}
   \mathbf{z}_t = \sigma_s(\mathbf{W}_z \mathbf{x}_t + \mathbf{U}_z \mathbf{h}_{t-1})
\end{equation}
%
The candidate activation is computed as:
%
\begin{equation}
\label{eq:gru-cand}
   \mathbf{\tilde{h}}_t = \sigma(\mathbf{W} \mathbf{x}_t + \mathbf{U}(\mathbf{r}_t \odot \mathbf{h}_{t-1}))
\end{equation}
%
The reset gate $\mathbf{r_{t}}$ determines how much of the current
input $\mathbf{x_{t}}$ is mixed in the previous state
$\mathbf{h}_{t-1}$ to form the candidate activation:
%
\begin{equation}
\label{eq:gru-reset}
   \mathbf{r}_t = \sigma_s(\mathbf{W}_r \mathbf{x}_t + \mathbf{U}_r \mathbf{h}_{t-1})
\end{equation}

By applying the $\mathrm{gru}$ function repeatedly a GRU layer maps a
sequence of inputs to a sequence of states:
\begin{equation}
  \mathrm{GRU}(\mathbf{X}, \mathbf{h}_0) = \mathrm{gru}(\mathbf{x}_n, \dots, \mathrm{gru}(\mathbf{x}_2, \mathrm{gru}(\mathbf{x}_1, \mathbf{h}_0)))
\end{equation}
where $\mathbf{X}$ stands for the matrix composed of input column vectors
$\mathbf{x}_1, \ldots, \mathbf{x}_n$. Two or more GRU layers can be composed into a stack: 
\begin{equation}
\mathrm{GRU}_2(\mathrm{GRU}_1(\mathbf{X}, {\mathbf{h_1}}_{0}), {\mathbf{h_2}}_{0}).
\end{equation}
In our version of the Stacked GRU architecture we use {\it residualized} layers:
\begin{equation}
\mathrm{GRU_{res}}(\mathbf{X}, \mathbf{h}_0) = \mathrm{GRU}(\mathbf{X}, \mathbf{h}_0) + \mathbf{X}
\end{equation}
Residual convolutional networks were introduced by
\newcite{he2015deep}, while \newcite{oord2016pixel} showed their
applicability to recurrent architectures. We adopt residualized layers
here as we observed they speed up learning in stacks of several
GRU layers.

Our gated recurrent units use steep sigmoids for gate activations: \[
\sigma_s(z) = \frac{1}{1 + \exp(-3.75z)} 
\]
and rectified linear units clipped between 0 and 5 for the unit
activations:
\[
\sigma(z) = \mathrm{clip(0.5(z+\mathrm{abs}(z)), 0, 5)}
\]

There are two more components of our {\sc Phon GRU} model: the
phoneme encoding layer, and mapping from the final state of the top GRU
layer to the image feature vector.
The phoneme encoding layer is a simply a lookup table $\mathbf{E}$ whose
columns correspond to one-hot-encoded phoneme vectors. The input
phoneme $p_t$ of utterance $p$ at each step $t$ indexes into the
encoding matrix and produces the input column vector:
\begin{equation}
  \mathbf{x}_t = \mathbf{E}[:,p_t].
\end{equation}
Finally, we map the final state of the top GRU layer ${\mathbf{h_K}}_n$
to the vector of image features using a fully connected layer:

\begin{equation}
  \hat{\mathbf{i}} = \mathbf{I} {\mathbf{h_K}}_n
\end{equation}

Our main interest lies in recurrent phoneme-level modeling. However, in order to
put the performance of the phoneme-level {\sc Phon GRU} into
perspective, we compare it to two word-level models. Importantly,
the word models should {\bf not} be seen as baselines, as they have access to
word boundary and word identity information not available to
{\sc Phon GRU}. 

\subsection{Word GRU}
The architecture of this model is the same as {\sc Phon GRU} with
the difference that we use words instead of phonemes as input symbols,
use learnable word embeddings instead of fixed one-hot phoneme
encodings, and reduce the number of layers in the GRU stack. See
Section~\ref{sec:experiments} for details.
\subsection{Word Sum}
The second model we use for comparison is a word-based non-sequential
model, consisting of a word embedding matrix, a vector sum operator,
and a mapping to the image feature vector:
\begin{equation}
  \label{eq:sum}
  \hat{\mathbf{i}} = \mathbf{I} \sum_{t=1}^n \mathbf{E}[:,w_t]
\end{equation}
where $w_t$ is the word at position $t$ in the input utterance.
This model simply learns word embeddings which are then summed into a
single vector and projected to the target image vector. Thus this model does
not have access to word sequence information, and is a distributed
analog of a bag-of-words model.

\section{Experiments}
\label{sec:experiments}

For all experiments, the models were trained on the training set of MS COCO. Textual input for the {\sc Phon GRU} models was transcribed automatically using the grapheme-to-phoneme functionality with the default English voice of the eSpeak speech synthesis toolkit.\footnote{Available at \url{http://espeak.sourceforge.net}} Stress and pause markers were removed, as well as word boundaries (after storing their position for use in experiments), leaving only phoneme symbols. See Figure~\ref{fig:ipa} for an example transcription.

\begin{figure}
  \centering
  \begin{tabular}{|l|}\hline
    \includegraphics[scale=0.7]{woman-ipa.png} \\
    A young woman riding a horse holding a flag\\\hline
  \end{tabular}
  \caption{Example of a postprocessed phonetic transcription output
    from eSpeak used as input to the {\sc Phon GRU} model.}
  \label{fig:ipa}
\end{figure}

Visual input for all models was obtained by forwarding images through the 16-layer convolutional neural network described in \newcite{simonyan2014very} pre-trained on Imagenet \cite{ILSVRCarxiv14}, and recording the activation vectors of the pre-softmax layer. The z-score transformation was applied to these features to ease optimization. 

Most of the details of the three model types and training hyperparameters were adopted from related work, and adapted via informal exploration. Full exploration of the search space was unfeasible due to the large number of adjustable settings in these models and their long running time. Given the importance of depth for our purposes, we did systematically explore the number of layers in the {\sc Phon GRU} and {\sc Word GRU} models. A single layer is optimal for {\sc Word GRU}. For {\sc Phon GRU}, see Section~\ref{subsec:visual} below. Other important settings were as follows:
\begin{itemize}
\item All models: Implemented in Theano \cite{Bastien-Theano-2012}, optimized with 
  Adam \cite{DBLP:journals/corr/KingmaB14}, initial learning rate of 0.0002, minibatch size
  of 64, gradient norm clipped to 5.0.
\item {\sc Word Sum}: 1024-dimensional word embeddings, words with frequencies below 10 replaced by {\tt UNK} token.
\item {\sc Word GRU}: 1024-dimensional word embeddings, a single 1024 dimensional hidden layer, words with frequencies below 10 replaced by {\tt UNK} token.
\item {\sc Phon GRU}: 1024-dimensional hidden layers.
\end{itemize}

\subsection{Prediction of visual features}
For the experiments on the prediction of visual features all models
were trained on the training set of MS COCO. As validation data we
used a random sample of 5000 images from the MS COCO validation set. 

Figure~\ref{fig:loss} shows the value of the validation average cosine distance
between the predicted visual vector and the target vector for three
random initializations of each of the model types. 

The Phonetic GRU model is more sensitive to the initialization: one
can clearly distinguish three separate trajectories. The word-level models
are much less affected by random initialization. In terms of the
overall performance, the {\sc Phoneme GRU} model falls between the
{\sc Word Sum} model and th {\sc Word GRU} model.

\begin{figure}
  \centering
  \includegraphics[scale=0.3]{loss-zoom.pdf}
  \caption{Value of the loss function on validation data during
    training. Three random initialization of each model are shown.}
  \label{fig:loss}
\end{figure}

We also evaluated the models on how well they perform when used to
search images: for each validation sentence the model was used to predict the
visual vector. The image vectors in the validation data were then
ranked by cosine similarity to the predicted vector, and the
proportion of times the correct image was among the top 3 was
reported. By {\it correct} image we mean the one which the sentence
was used to describe (even though often many other images are also
good matches to the sentence). 

In Figure~\ref{fig:accat5} we report the accuracies on this task for
the two word-level models, as well as for the Phoneme GRU model with
different number of hidden layers. We trained each model version with
three random initializations for each model setting, and evaluate
after each epoch. We report the score of the best epoch for each
initialization. The overall ranking of the models matches the direct
evaluation of the loss function above: the phoneme-level models are in
between the two word-level models. Phoneme GRU with three hidden
layers is the best of the phoneme-level model. This model achieved the
maximum score after 8 epochs. In subsequent experiments we use these
particular model parameters when reporting other results with {\sc
  Phoneme GRU}.

\begin{figure}
  \centering
  \includegraphics[scale=0.25]{accat5.pdf}
  \caption{Accuracy at 5 on the image retrieval task}
  \label{fig:accat5}
\end{figure}

% \begin{table}
%   \centering
%   \begin{tabular}{ll|r}
%  Model     & Layers & Acc. at 5 \\\hline
%  Word Sum  & 0      & 0.158 \\
%  Word GRU  & 1      & 0.205 \\\hline
%  Phon GRU  & 1      & 0.169  \\
%  Phon GRU  & 2      & 0.176  \\
%  Phon GRU  & 3      & \bf 0.183  \\  
%  Phon GRU  & 4      & 0.179 \\
%  Phon GRU  & 5      & 0.170 \\
%   \end{tabular}
% \caption{Accuracy at 5 on the image retrieval task.}
% \label{tab:accat5}
% \end{table}

\subsection{Word boundary prediction}
To explore the sensitivity of the {\sc Phon GRU} model to linguistic structure at the sub-word level, we investigated the encoding of information about word-ends in the hidden layers. A logistic regression model was trained on activation patterns of the hidden layers at all timesteps, with the objective of identifying phonemes that preceded a word boundary. For comparison, we also trained logistic regression models on \textit{n}-gram data to perform the same tasks, with positional phoneme \textit{n}-grams in the range 1-\textit{n}. Both the \textit{n}-gram models and hidden state models were implemented using Scikit-learn \cite{scikit-learn} {\tt LogisticRegression} implementation with L2-regularization and the $C$ parameter set to 1000. The location of the word boundaries was taken from the eSpeak transcriptions, which mostly matches the location of word boundaries according to conventional English spelling. However, eSpeak models some coarticulation effects which sometimes leads to word boundaries disappearing from the transcription. For example, {\it bank of a river} is transcribed as \textipa{[baNk @v@ \*rIv@]}.
%[example of a phonetic transcription with 'ova']

Accuracy scores reported in Table~\ref{tab:boundary} are the averages over 5-fold cross-validation on the captions of 5000 images from the validation portion of MS COCO. The proportion of phonemes preceding a word boundary is 0.29, meaning that predicting {\it no word boundary} by default would be correct in 0.71 of cases. At the highest hidden layer, enough information about the word form is available to correctly predict word end in 0.82 of cases - substantially above the majority baseline. The lower levels allow for more accurate prediction of word boundaries: 0.86 at the middle hidden layer, and 0.88 at the bottom level. 
Prediction accuracy of the logistic regression model based on the activation patterns of the lowest hidden layer is comparable to that of a bigram logistic regression model.

These results indicate that information on sub-word structure is only partially encoded by {\sc Phon GRU}, and is mostly absent by the time the signal from the input propagates to the top layer. The bottom layer does learn to encode a fair amount of word boundary information, but the prediction score substantially below 100\% indicates that it is rather selective. 

\begin{table}[]
	\centering
	\begin{tabular}{lrr}
		Model & & Accuracy \\\hline
                Majority & & 0.71 \\
		\hline
		Activation vector & Layer 1 & 0.88 \\
		& Layer 2 & 0.86 \\
		& Layer 3 & 0.82 \\
		\hline
		\textit{n}-gram & \textit{n} = 1 & 0.79 \\
		& \textit{n} = 2 & 0.88 \\
		& \textit{n} = 3 & 0.93 \\
		& \textit{n} = 4 & 0.95
	\end{tabular}
	\caption{Accuracy of word boundary prediction}
\label{tab:boundary}
\end{table}

\subsection{Word similarity}
To understand the encoding of semantic information in {\sc Phon GRU}, we analyzed the cosine similarity of activation vectors for word pairs from the MEN dataset \cite{bruni2014multimodal}, and compared them to human similarity judgements.
For each word pair in the MEN dataset, the words were transcribed phonetically using eSpeak and then fed to {\sc Phon GRU} individually. For comparison, the words were also fed to {\sc Word GRU} and {\sc Word Sum}. Word pair similarity was quantified as the cosine similarity between the activation patterns of the hidden layers at the end-of-sentence symbol.
In contrast to {\sc Word GRU} and {\sc Word Sum}, {\sc Phon GRU} has access to the sub-word structure. To explore the role of phonemic form in word similarity, a measure of phonemic difference was included: the Levenshtein distance between the phonetic transcriptions of the two words, normalized by  the length of the longer transcription. 

Table~\ref{tab:human} shows Spearman's rank correlation coefficient between human similarity ratings from the MEN dataset and cosine similarity at the last timestep for all hidden layers. In all layers, the cosine similarities between the activation vectors for two words are significantly correlated with human similarity judgements. The strength of the correlation differs considerably between the layers, ranging from 0.09 in the first layer to 0.28 in the highest hidden layer. The second column in Table~\ref{tab:human} shows the correlations when only taking into account the 1283 word pairs of which both words appear at least 100 times in the training data. 
Correlations for both {\sc Word GRU} and {\sc Word SUM} are considerably higher than for {\sc Phon GRU}. This is expected given that these are word level models with explicit word-embeddings, while {\sc Phon GRU} builds word representations by forwarding phoneme-level input through several layers of processing.

\begin{table*}[]
	\centering
	\begin{tabular}{rrr}
                          & All words & Frequent words \\\hline
{\sc Phon GRU} Layer 1 & 0.09 & 0.12\\
		  Layer 2 & 0.21 & 0.33 \\
		  Layer 3 & 0.28 & 0.45 \\
		\hline
{\sc Word GRU} & 0.48 & 0.60\\	\hline
{\sc Word Sum} & 0.42 & 0.56
	\end{tabular}
	\caption{Spearman's correlation coefficient between
          word-word cosine similarity and human similarity judgements. All
          correlations significant at \textit{p} $< 0.01$. Frequent
          words appear at least 100 times in the training data.} 
\label{tab:human}
\end{table*}

Table~\ref{tab:edit} shows Spearman's rank correlation coefficient between the edit distance and the cosine similarity of activation vectors at the hidden layers of {\sc Phon GRU}.
As expected, edit distance and cosine similarity of the activation vectors are negatively correlated, which means that words which are more similar in form are also more similar according to the model. (Note that in the MEN dataset, meaning and word form are also (weakly) correlated: human similarity judgements and edit distance are correlated at -0.08 (\textit{p} $<$ 0.05).)

The negative correlation between edit distances and cosine similarities is strongest at the lowest hidden layer and weakest, though still present and stronger than for human judgements, at the third hidden layer. 

The correlations of cosine similarities with edit distance on the one hand, and human similarity rating on the other hand, indicate that the different hidden layers reflect increasing levels of representation: whereas at the lowest level mostly encodes information about form, the highest layer mostly encodes semantic information.

\begin{table}[h]
	\centering
	\begin{tabular}{rr}
                Layer   & $\rho$ \\\hline
		      1 & $-0.30$ \\
		      2 & $-0.24$ \\
		      3 & $-0.15$
	\end{tabular}
	\caption{Spearman's rank correlation coefficient between
          {\sc Phon GRU} cosine similarity and phoneme-level edit distance. All correlations significant at \textit{p} $< 0.01$.} 
\label{tab:edit}
\end{table}

\subsection{Position of shared substrings}
The purpose of this experiment is to quantify the time-scale at which information is retained in the different layers of {\sc Phon GRU}. We thus looked at the location of phoneme strings shared by sentences and their nearest neighbours. Captions associated with 5000 images from the validation set of MS COCO were included in this experiment. We determined each sentences' nearest neighbour for each hidden layer in {\sc Phon GRU}. The nearest neighbour is the sentence for which the activation vector at the end of sentence symbol has the smallest cosine distance to the activation vector of the original sentence. The position of matching substrings is the average position in the original sentence of symbols in substrings that are shared by the neighbour sentences, counted from the end of the sentence. A high mean average substring position thus means that the shared substring(s) appear early in the sentence. This gives an indirect measure of the timescale at which the different layers operate. Figure~\ref{fig:example-shared} shows an example illustrating this idea. 

As can be seen in Table~\ref{tab:substrings}, the average position of shared substrings in neighbour sentences is closest to the end for the first hidden layer and moves towards the beginning of the sentence for the second and third hidden layer. This indicates a difference between the layers with regards to the timescale they represent. Whereas in the lowest layer only information from the latest timesteps is present, the higher layers retain the input  signal over longer timescales.


\begin{figure}[h]
  \begin{tabular}{c}
    Layer 1 \\\hline
    A metallic bench {\bf on a path in} the {\bf park} \\
    A man riding a bicycle {\bf on a path} in a {\bf park} \\\hline
    Layer 3 \\\hline
    A metallic {\bf bench} on a path in the {\bf park} \\
    A stone park {\bf bench} sitting in an empty green {\bf park}\\ \hline
  \end{tabular}
  \caption{An illustrative sentence with its nearest neighbor at layer 1 and layer 3. For readability, sentences are displayed in conventional spelling, and only highlight matching substrings of length $\geq3$. In reality we used phonetic transcriptions to compute shared substring positions, and used substrings of all lengths. }
\label{fig:example-shared}
\end{figure}


\begin{table}[h]
	\centering
	\begin{tabular}{rr}
                Layer   & Mean position \\\hline
		      1 & 12.1 \\
		      2 & 14.9 \\
		      3 & 16.8 \\
	\end{tabular}
	\caption{Average position of symbols in shared substrings
          between nearest neighbour sentences according to Phon GRU
          representations at the different layers. Positions are indexed from
          end of string, i.e. index 0 is the last symbol.}
\label{tab:substrings}
\end{table}
\section{Discussion}
\label{sec:discussion}
In this paper we have shown that a model of stacked Gated Recurrent Units is capable of learning to extract visually significant aspects of meaning from sequential phoneme data. The {\sc Stacked GRU} model was trained to predict a high level image feature vector from phonetically transcribed captions. On the training image retrieval task, {\sc Stacked GRU} was outperformed by a single hidden layer GRU model with a word embedding layer that takes words as input. However, accuracy of {\sc Stacked GRU} on the training task was higher than that of {\sc Word Sum}, the analog of a bag of words model. This indicates that although taking phoneme-level instead of word-level data as input makes the training task more difficult, the GRU-architecture does allow {\sc Stacked GRU} to exploit sentence order information.
We explored the role of each of the layers in the stack of hidden units in {\sc Stacked GRU} in representing form as well as meaning. A word boundary prediction experiment indicated that the lowest layer is most involved in encoding information about word form, as its activation pattern provided a mre accurate predictor of word boundaries than that of the second, and even more so for the third layer. A word similarity experiment on the MEN dataset showed that cosine similarity at any hidden layer and edit distance of two word pairs were negatively correlated. This correlation is strongest in the lowest hidden layer and decreases in magnitude for the higher layers. 
Human judgements of semantic relatedness were correlated most strongly with cosine similarities between the highest hidden layer, less strongly with cosine similarities of the middle hidden layer, and weakest in the lowest hidden layer, indicating that semantic information is encoded mostly in the higher layers. On the word similarity judgement task, the cosine distances between activation patterns of the hidden layer of both word-based models were correlated more strongly with human similarity judgements than those of {\sc Stacked GRU}. This may be due to the fact that these models have a word embedding layer, and most likely is also helped by the fact that these models do not have the additional task of recognizing a word as such. [suggest replacement experiment here]
As we go up in the stack of hidden layers, the timescale on which the layer operates increases. Sentences that have similar activation vectors at the end of sentence in the highest hidden layer have shared sequences at positions earlier in the sentence than sentences that are similar in the second, and certainly than in the first layer. These findings are consistent with \newcite{hermans2013training}. It shows the input data is processed hierarchically through time, with the lowest layer capturing structure over short sequences, such as words, and the higher levels capturing structure over longer stretches, such as sentences. 

[suggest that it may be interesting to investigate the role of word form more thoroughly, perhaps with regards to morphology, or when encountering unknown words]



\bibliographystyle{emnlp2016}
\bibliography{biblio}

\end{document}